\documentclass[10pt]{article}

\usepackage[margin=1in]{geometry}
\PassOptionsToPackage{hyphens}{url}\usepackage{hyperref}
\usepackage{enumitem}
\usepackage{multicol}
\usepackage{color}
\usepackage{fancyvrb}
\usepackage{textpos}
\usepackage[font=small,labelfont=bf]{caption}
\usepackage{amsmath}
\usepackage{amssymb}
\usepackage{amsthm}
\usepackage{lastpage}
\usepackage{fancyhdr}
\pagestyle{fancy}
\usepackage{setspace}
\usepackage{etoolbox}
\usepackage{mathrsfs}
\AtBeginEnvironment{quote}{\singlespacing\small}
%\usepackage{pxfonts}
%\usepackage{lmodern}
%\usepackage{txfonts}
%\usepackage{newtxmath}
\usepackage{newtxtext}
%\usepackage{newpxtext}
\usepackage{multirow}

\usepackage{listings}
\lstset{%
    language=C,
    basicstyle=\ttfamily,
    keywordstyle=\bfseries,
    showstringspaces=false,
    morekeywords={for, if, do, then},
    mathescape
}%

\usepackage{tikz}
\usetikzlibrary{positioning}
\usetikzlibrary{automata}
\tikzset{>=stealth}

\newenvironment{myitemize}{\begin{itemize}\setlength\itemsep{1em}}{\end{itemize}}
\newenvironment{myenum}{\begin{enumerate}\setlength\itemsep{1em}}{\end{enumerate}}

\newcommand{\mh}[1]{\textcolor{magenta}{\textbf{#1}}}
\newcommand{\rh}[1]{\textcolor{red}{\textbf{#1}}}
\newcommand{\gh}[1]{\textcolor{green}{\textbf{#1}}}
\newcommand{\ch}[1]{$\mathtt{#1}$}
\newcommand{\uu}[1]{\underline{\underline{#1}}}

\newtheorem{theorem}{Theorem}[section]
\newtheorem{lemma}{Lemma}[section]
\newtheorem{corollary}{Corollary}[section]
\newtheorem{fact}{Fact}[section]
\newtheorem{definition}{Definition}[section]
\newtheorem{proposition}{Proposition}[section]
\newtheorem{observation}{Observation}[section]
\newtheorem{claim}{Claim}[section]
\newtheorem{remark}{Remark}[section]
\newtheorem{conjecture}{Conjecture}[section]

\setlength{\textheight}{9.5 in}
\setlength{\textwidth}{6.75 in}
\setlength{\oddsidemargin}{-0.125 in}
\setlength{\evensidemargin}{-0.125 in}
\setlength{\topmargin}{-1.0 in}
\setlength{\parskip}{3pt}

\pagestyle{fancy}

\fancyhf{}
\cfoot{Page \thepage\ of \pageref{LastPage}}

\begin{document}

\begin{center}
  \textbf{The Red-Black Tree Data Structure} \\
  \textbf{Algorithms and Implementation}
\end{center}

\section{Introduction and Background}

Let $S$ be a data structure, then we say that $S$ contains a collection
of $nodes$ and supports a set of $operations$ defined on the nodes of the
structure.

Let $x$ be a node, then we assume that $x$ contains a field, $x.k$
representing the key associated to that node, and may also include further
fields as necessary (i.e. satellite data). Keys must be from a totally
ordered universe, and are not necessarily unique.

\begin{definition}
A \textbf{dynamic set} data structure, $S$, supports the following operations:
\begin{enumerate}
    \setlength\itemsep{1em}
    \item $search(S, k)$. Search for an return a pointer to a node in $S$ with
      key $k$, or report that no such node exists.
    \item $insert(S, k)$. Add a new node to $S$ with key $k$.
    \item $delete(S, k)$. Remove a node in $S$ with key $k$, if possible.
    \item $minimum(S, k)$. Find and return a pointer to a node in $S$
      with minimal key value.
    \item $maximum(S,k)$. Find and return a pointer to a node in $S$ with
      maximal key value.
    \item $predecessor(S, k)$. Find a return a pointer to a node in $S$ with
      key that is the greatest value less than $k$.
    \item $successor(S, k)$. Find a return a pointer to a node in $S$ with
      key that is the least value greater than $k$.
\end{enumerate}
\end{definition}

\begin{definition}
  A \textbf{dictionary} is a data structure that supports the dynamic set
  operations of $search$, $insert$, and $delete$.
\end{definition}

One fundamental example of a data structure that supports all of the
operations of a dynamic set is the \textbf{binary search tree} (BST). We will
not review the details of this structure here, but only note a few important
facts.

Let $T$ be a BST, and let $h = height(T)$. Then, the dynamic set operations
all run in time $O(h)$. We thus want $h$ to be as small as possible as a
function of the number of nodes, $n$, of the tree.

What can go wrong? In the worst case, the height of a BST may be $n$, i.e. a
single chain of nodes resembling a linked list. Therefore, in the worst case,
the dictionary operations may take $O(n)$ time. Indeed, no better than a linked list.

There have been many variants of the basic BST data structure that aims
to ensure that $T$ is \textbf{balanced} in the sense that no one path is
much longer than any other. Different trees support different balancing schemes,
e.g. AVL trees, 2-3 trees, and so on. We will focus on the \textbf{red-black}
tree.

\section{Red-Black Trees}

A red-black tree is fundamentally a binary search tree with
some additional structure: a dichromatic coloring scheme, and some further
restrictions.

The nodes of a red-black tree will contain
the following fields:
\begin{itemize}
  \item $k$, the key of the node.
  \item $left$, a pointer to the left child of the node.
  \item $right$, a pointer to the right child of the node.
  \item $parent$, a pointer to the node's parent.
  \item $color$, the color of the node.
\end{itemize}

Let $T$ be a tree (or, indeed, any graph), then $T(V)$ is the set of nodes
of the tree and $T(E)$ is the set of edges.

\begin{definition}
  A \textbf{red-black tree} (RBT) is a BST, $T$, that satisfies the following additional
  properties:
  \begin{enumerate}[label=\Roman*.]
    \item $\forall v \in T(V) \; v.color \in \{\texttt{RED}, \texttt{BLACK}\}$,
    \item $T.root.color = \texttt{BLACK}$,
    \item Every leaf node is colored \texttt{BLACK},
    \item $\forall v \in T(V)$ if $v.color = \texttt{RED}$,
      then $v.left.color = v.right.color = \texttt{BLACK}$,
    \item $\forall v \in T(V)$ all simple paths from $v$ to descedant leaves
      contain the same number of black nodes.
  \end{enumerate}
\end{definition}

Properties $I,\ldots, V$ are refered to as RB-I,...,RB-V below.

\begin{definition}
  Let $T$ be an RBT and suppose $v \in T(V)$. The \emph{black-height} of
  $v$ is given by:
    \[ bh(v) := \#\text{black nodes on any simple path from x down to a leaf,
      excluding v itself.} \]
  Additionally, $bh(T) := bh(T.root)$, the black-height of the root node.
\end{definition}

\begin{claim}
  Let $T$ be an RBT and $v \in T(V)$, then $bh(v)$ is \emph{well-defined}.
\end{claim}

\begin{proof}
  By RB-V, all paths from $v$ to any descendant leaves
  have the same number of black nodes.
  Thus $bh : T(V) \rightarrow \mathbb{N}$ is indeed a well-defined function.
\end{proof}

\end{document}
